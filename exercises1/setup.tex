%\usepackage{algorithm}
%\usepackage{algorithmic}
%\usepackage{amsmath}
%\usepackage{amssymb}
%\usepackage{array}
%\usepackage{blindtext}
\usepackage{booktabs}
%\usepackage{colortbl}
%\usepackage{comment}
%\usepackage{etex}
%\usepackage{float}
%\usepackage[T1]{fontenc}
%\usepackage{graphicx}
%\usepackage[space]{grffile}
%\usepackage{multirow}
%\usepackage[utf8]{inputenc}
%\usepackage{pgfplots}
%\usepackage{pgfplotstable}
\usepackage{subcaption}
%\usepackage{standalone}
\usepackage{times}
\usepackage{tikz}
%\usepackage{todonotes}
%\usepackage{xspace}
\usepackage[breaklinks=true,colorlinks,bookmarks=false]{hyperref} %Hyperref should go last

%float setup
%\newfloat{algorithm}{t}{}
%
%%hyperref setup
\hypersetup{colorlinks=true}
\hypersetup{linktoc=all}
\hypersetup{draft=false}
\hypersetup{urlcolor=blue}
\hypersetup{citecolor=black}
\hypersetup{linkcolor=black}
%
%%tikz setup
%\usetikzlibrary{arrows,calc,external,fit,shapes}
%\usetikzlibrary{backgrounds}
%\usetikzlibrary{pgfplots.groupplots}
%
%\pgfplotsset{compat=1.5.1}
%\pgfplotsset{filter discard warning=false}
%
%%Set graphics path
%\graphicspath{{figures//}}
%
%\DeclareMathOperator*{\argmin}{arg\,min}
%\DeclareMathOperator*{\argmax}{arg\,max}
%
%%Make require and ensure input and output
%\renewcommand{\algorithmicrequire}{\textbf{Input:}}
%\renewcommand{\algorithmicensure}{\textbf{Output:}}
%
%%This can be used to add external file dependencies for latexmk
%\makeatletter
%\newcommand*{\addFileDependency}[1]{% argument=file name and extension
%  \typeout{(#1)}% latexmk will find this if $recorder=0 (however, in that case, it will ignore #1 if it is a .aux or .pdf file etc and it exists! if it doesn't exist, it will appear in the list of dependents regardless)
%  \@addtofilelist{#1}% if you want it to appear in \listfiles, not really necessary and latexmk doesn't use this
%  \IfFileExists{#1}{}{\typeout{No file #1.}}% latexmk will find this message if #1 doesn't exist (yet)
%}
%\makeatother
%
\newcommand{\mytodo}[1]{\textcolor{red}{TODO: #1}}
